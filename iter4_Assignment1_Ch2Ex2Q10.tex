\documentclass[journal,12pt,twocolumn]{IEEEtran}
\usepackage{tikz}
\usepackage{amsmath}
\usepackage{amssymb}
\pagestyle{empty}
\usepackage{setspace}
\usepackage{gensymb}
\singlespacing

\usepackage{amsmath}
\usepackage{amsthm}
\begin{document}
\newcommand{\myvec}[1]{\ensuremath{\begin{pmatrix}#1\end{pmatrix}}}
\newcommand{\cmyvec}[1]{\ensuremath{\begin{pmatrix*}[c]#1\end{pmatrix*}}}
\newcommand{\mydet}[1]{\ensuremath{\begin{vmatrix}#1\end{vmatrix}}}
\newcommand{\proj}[2]{\textbf{proj}_{\vec{#1}}\vec{#2}}
\let\StandardTheFigure\thefigure
\let\vec\mathbf

\title{
Assignment - 1
}
\author{ Prasanna Kumar R - SM21MTECH14001}
\maketitle
\newpage
\bigskip
\bibliographystyle{IEEEtran}
\section*{\textbf{Problem}}
\noindent
\textbf{An isoceles triangle has the extremities of its base at \myvec{2\\5} and \myvec{-2\\2}.
Find the two possible positions of the vertex if its area is 25 sq.units} 


\begin{tikzpicture}
[scale=2,>=stealth,point/.style={draw,circle,fill = black,inner sep=0.5pt},]

%Triangle sides
\def\a{6}
\def\b{5}
\def\c{5}
 
%Coordinates of A
%\def\p{{\a^2+\c^2-\b^2}/{(2*\a)}}
%\def\p{2.25}
%\def\q{{sqrt(\c^2-\p^2)}}

%Labeling points
\node (A) at (0,1.5)[point,label=above right:$A$] {};
\node (B) at (-1.5, 0)[point,label=below left:$B$] {};
\node (C) at (1.5, 0)[point,label=below right:$C$] {};

%Foot of perpendicular

\node (D) at (0,0)[point,label=above right:$D$] {};

%Drawing triangle ABC
\draw (A) -- node[left] {$\textrm{b}$} (B) -- node[below] {$\textrm{}$} (C) -- node[above,xshift=2mm] {$\textrm{b}$} (A);

%Drawing altitude AD
\draw (A) -- node[left] {$\textrm{h}$}(D);

%Drawing and marking angles
%\tkzMarkAngle[fill=orange!40,size=0.5cm,mark=](A,C,B)
%\tkzMarkAngle[fill=orange!40,size=0.4cm,mark=](D,B,A)
%\tkzMarkAngle[fill=green!40,size=0.5cm,mark=](B,A,C)
%\tkzMarkAngle[fill=green!40,size=0.5cm,mark=](C,B,D)
%tkzMarkRightAngle[fill=blue!20,size=.2](A,D,B)
%\tkzMarkRightAngle[fill=blue!20,size=.2](B,D,A)
%\tkzLabelAngle[pos=0.65](A,C,B){$\theta$}
%\tkzLabelAngle[pos=0.65](A,B,D){$\theta$}
%\tkzLabelAngle[pos=1](B,A,C){\rotatebox{-45}{$\alpha = 90\degree -\theta$}}
%\tkzLabelAngle[pos=0.65](C,B,D){$\alpha$}

\end{tikzpicture}

\noindent
\section*{\textbf{Solution}}
\noindent
Let the vertex B be \myvec{2\\5} and vertex C be \myvec{-2\\2}.\\[6pt]
Let the other vertex A be $\myvec{x\\y}$.\\[6pt]
Let us find the distance of BC
\begin{center}
$\vec{B} = \myvec{2\\5}, \vec{C} =\myvec{-2\\2}$
\end{center}
$$\left\|\mathbf{B-C}\right\|=\left\|\mathbf{\myvec{2\\5}-\myvec{-2\\2}}\right\|=\sqrt{4^2+3^2}=5$$
Given, the area of the triangle= 25 sq.units\\
We know area of a $\triangle$ with the vertices A,B and C
can be given by:

$$ \label{eq:area_tri}
\mathbf{\Delta =\frac{1}{2}\begin{vmatrix}
1 & 1 & 1\\ 
A & B & C
\end{vmatrix}}$$
$$\Delta ABC=\frac{1}{2}\begin{vmatrix}
1 & 1 & 1\\ 
x & 2 & -2\\ 
y & 5 & 2
\end{vmatrix}$$
Expanding along the first row,\\
$$\Delta ABC=\frac{1}{2}\left [ 1(4+10)-1(2x+2y)+1(5x-2y) \right ]$$
$$\frac{1}{2}\left [14-(2x+2y)+(5x-2y) \right ]=25$$
$$-2x-2y+5x-2y=50-14$$
\begin{equation}
 3x-4y=36    
\end{equation}
\noindent
Since ABC is an Isoceles triangle, \\ AD is the perpendicular bisector of BC
\begin{equation}
 (\vec{A}-\vec{D})^T (\vec{B}-\vec{C})=0
\end{equation}
Also,
\begin{align*}
    \vec{D} &=\frac{1}{2}(\vec{A}+\vec{B}) \\[6pt]
    \vec{D} &=\frac{1}{2}\left[ \myvec{2\\5} + \myvec{-2\\2}\right] \\[6pt]
    \vec{D} &=\myvec{0\\3.5} 
\end{align*}
Now,
\begin{align*}
    (\vec{A}-\vec{D}) &= \myvec{x\\y}- \myvec{0\\3.5} \\[6pt]
    (\vec{A}-\vec{D}) &= \myvec{x\\y-3.5} \\[6pt]
    (\vec{B}-\vec{C}) &= \myvec{2\\5}- \myvec{-2\\2} \\[6pt]
    (\vec{B}-\vec{C}) &= \myvec{4\\3}
\end{align*}
By (2),
\begin{align*}
{\myvec{x\\y-3.5}}^T \myvec{4\\3} &= 0 \\[6pt]
    4x+3(y-3.5) &= 0 \\[6pt]
    4x+3y &= 10.5
\end{align*}
Multiply by 2,
\begin{equation}
 \implies 8x+6y=21    
\end{equation}
\noindent
\newpage
\noindent
By interchanging the columns, the area of triangle can also be expressed as,
$$\Delta ABC=\frac{1}{2}\begin{vmatrix}
1 & 1 & 1\\ 
x & -2 & 2\\ 
y & 2 & 5
\end{vmatrix}$$
\noindent
Expanding along the first row,
\begin{align*}
\frac{1}{2}\left [ 1(-10-4)-1(5x-2y)+1(2x+2y) \right ] &= 25 \\
\frac{1}{2}\left [-14-1(5x-2y)+1(2x+2y) \right ] &= 25 \\
-5x+2y+2x+2y &= 64
\end{align*}
 \begin{align}
    \implies  -3x+4y &= 64   
 \end{align}
Let us consider the matrices representation of equations (1) and (3),
\begin{align*}
\begin{bmatrix}
3 & -4 \\
8 & 6 
\end{bmatrix}
\begin{bmatrix}
x \\ y
\end{bmatrix}
& =
\begin{bmatrix}
36 \\ 21
\end{bmatrix} \\[6pt]
  \vec{A}\vec{X} &=\vec{B} \\[6pt]
  \vec{X} &= \vec{A}^{-1}\vec{B} \\[6pt]
\begin{bmatrix}
x \\ y
\end{bmatrix}
& =
\begin{bmatrix}
3 & -4 \\
8 & 6 
\end{bmatrix}^{-1}
\begin{bmatrix}
36 \\ 21
\end{bmatrix}  \\[6pt]
\begin{bmatrix}
x \\ y
\end{bmatrix}
& =
\frac{1}{50}
\begin{bmatrix}
6 & 4 \\
-8 & 3 
\end{bmatrix}
\begin{bmatrix}
36 \\ 21
\end{bmatrix} \\[6pt]
\begin{bmatrix}
x \\ y
\end{bmatrix}
& =
\frac{1}{50}
\begin{bmatrix}
300 \\ -225
\end{bmatrix}
\end{align*}
\begin{equation}
\begin{bmatrix}
x \\ y
\end{bmatrix}
=
\begin{bmatrix}
6 \\ -4.5
\end{bmatrix}
\end{equation}
\noindent
Let us consider the matrices representation of equations (4) and (3),
\begin{align*}
\begin{bmatrix}
-3 & 4 \\
8 & 6 
\end{bmatrix}
\begin{bmatrix}
x \\ y
\end{bmatrix}
& =
\begin{bmatrix}
64 \\ 21
\end{bmatrix} \\[6pt]
\vec{A}\vec{X} &=\vec{B} \\[6pt]
\vec{X} &= \vec{A}^{-1}\vec{B} \\[6pt]
\begin{bmatrix}
x \\ y
\end{bmatrix}
& =
\begin{bmatrix}
-3 & 4 \\
8 & 6 
\end{bmatrix}^{-1}
\begin{bmatrix}
64 \\ 21
\end{bmatrix} \\[6pt]
\begin{bmatrix}
x \\ y
\end{bmatrix}
& =
\frac{1}{-50}
\begin{bmatrix}
300 \\ -575
\end{bmatrix}
\end{align*}
\begin{equation}
\begin{bmatrix}
x \\ y
\end{bmatrix}
=
\begin{bmatrix}
-6 \\ 11.5
\end{bmatrix}
\end{equation}
\\\\
\noindent
$\therefore$ From equations (5) and (6), \textbf{The two possible vertex are $(6,-4.5)$ and $(-6,11.5)$}
\end{document}