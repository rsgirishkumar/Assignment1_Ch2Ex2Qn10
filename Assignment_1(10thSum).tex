\documentclass[journal,12pt,twocolumn]{IEEEtran}
\usepackage{tikz}
\usepackage{amsmath}
\usepackage{amssymb}
\pagestyle{empty}
\usepackage{setspace}
\usepackage{gensymb}
\singlespacing

\usepackage{amsmath}
\usepackage{amsthm}
\begin{document}
\newcommand{\myvec}[1]{\ensuremath{\begin{pmatrix}#1\end{pmatrix}}}
\newcommand{\cmyvec}[1]{\ensuremath{\begin{pmatrix*}[c]#1\end{pmatrix*}}}
\newcommand{\mydet}[1]{\ensuremath{\begin{vmatrix}#1\end{vmatrix}}}
\newcommand{\proj}[2]{\textbf{proj}_{\vec{#1}}\vec{#2}}
\let\StandardTheFigure\thefigure
\let\vec\mathbf

\title{
Assignment - 1
}
\author{ Prasanna Kumar R - SM21MTECH14001}
\maketitle
\newpage
\bigskip
\bibliographystyle{IEEEtran}
\section*{\textbf{Problem}}
\noindent
\textbf{An isoceles triangle has the extremities of its base at $(2,5)$ and $(-2,2)$.
Find the two possible positions of the vertex if its area is 25 sq.units} 


\begin{tikzpicture}
[scale=2,>=stealth,point/.style={draw,circle,fill = black,inner sep=0.5pt},]

%Triangle sides
\def\a{6}
\def\b{5}
\def\c{5}
 
%Coordinates of A
%\def\p{{\a^2+\c^2-\b^2}/{(2*\a)}}
%\def\p{2.25}
%\def\q{{sqrt(\c^2-\p^2)}}

%Labeling points
\node (A) at (0,1.5)[point,label=above right:$A$] {};
\node (B) at (-1.5, 0)[point,label=below left:$B$] {};
\node (C) at (1.5, 0)[point,label=below right:$C$] {};

%Foot of perpendicular

\node (D) at (0,0)[point,label=above right:$D$] {};

%Drawing triangle ABC
\draw (A) -- node[left] {$\textrm{b}$} (B) -- node[below] {$\textrm{}$} (C) -- node[above,xshift=2mm] {$\textrm{b}$} (A);

%Drawing altitude AD
\draw (A) -- node[left] {$\textrm{h}$}(D);

%Drawing and marking angles
%\tkzMarkAngle[fill=orange!40,size=0.5cm,mark=](A,C,B)
%\tkzMarkAngle[fill=orange!40,size=0.4cm,mark=](D,B,A)
%\tkzMarkAngle[fill=green!40,size=0.5cm,mark=](B,A,C)
%\tkzMarkAngle[fill=green!40,size=0.5cm,mark=](C,B,D)
%tkzMarkRightAngle[fill=blue!20,size=.2](A,D,B)
%\tkzMarkRightAngle[fill=blue!20,size=.2](B,D,A)
%\tkzLabelAngle[pos=0.65](A,C,B){$\theta$}
%\tkzLabelAngle[pos=0.65](A,B,D){$\theta$}
%\tkzLabelAngle[pos=1](B,A,C){\rotatebox{-45}{$\alpha = 90\degree -\theta$}}
%\tkzLabelAngle[pos=0.65](C,B,D){$\alpha$}

\end{tikzpicture}

\noindent
\section*{\textbf{Solution}}
\noindent
Let the vertex B be $(2,5)$ and vertex C be $(-2,2)$.\\[6pt]
Let the other vertex A be $(x,y)$.\\[6pt]
Let us find the distance of BC
\begin{center}
$\vec{B} = \myvec{2\\5}, \vec{C} =\myvec{-2\\2}$
\end{center}
$$\left\|\mathbf{B-C}\right\|=\left\|\mathbf{\myvec{2\\5}-\myvec{-2\\2}}\right\|=\sqrt{4^2+3^2}=5$$
Given, the area of the triangle= 25 sq.units\\
$$\displaystyle\frac{1}{2}*BC*AD= 25$$
$$\dfrac{1}{2}*5*h=25$$
$$AD=h=10$$
\noindent
Since ABC is an Isoceles triangle,
$$AB=AC$$
$$AB^2=AC^2$$
$$(x-2)^2+(y-5)^2=(x+2)^2+(y-2)^2$$
$$4x-4y+4=-4x-10y+25$$
\begin{equation}
    {8x+6y=21}
\end{equation} \\[6pt]
Point D is the Midpoint of BC \\[6pt]
$$D=\left(\frac{2-2}{2},\frac{5+2}{2}\right)$$
$$D=(0,3.5)$$
We know that $AD=h=10$
$$AD^2=100$$
\begin{equation}
  x^2+(y-3.5)^2=100
\end{equation}\\[6pt]
From (1), $x=\dfrac{21-6y}{8}$. Subs $x$ in equation (2)...
$$\left(\frac{21-6y}{8}\right)^2+(y-3.5)^2=100$$ \\[6pt]
Solving the above equation, \\[6pt]
$$1.5625y^2-10.9375y-80.86=0$$
$$y=11.5,-4.5$$\\[6pt]
When $y=11.5$ , $x=\dfrac{21-6(11.5)}{8}= -6$ \\[3pt]
When $y=4.5$ ,  $x=\dfrac{21+6(4.5)}{8}=6$ \\[12pt]
\textbf{Therefore, The two possible vertex are $(-6,11.5)$ and $(6,-4.5)$}
\end{document}